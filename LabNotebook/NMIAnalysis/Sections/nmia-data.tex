\section{The data}
	
	\subsection{Zernike coefficients dataset}
			8 datasets of zernike coefficients are created, each of the dataset contain 5000 datapoints
			\begin{itemize}
				\item \textbf{2 mode dataset}: 2 Zernike modes coefficients, their RMSE in the range [-1,1]
				\item \textbf{5 mode dataset}: 5 Zernike modes coefficients, their RMSE in the range:
					\begin{itemize}
						\item Modes 2 and 3: [-0.4, 0.4]
						\item Modes 4, 5 and 6: [-0.4, 0.4]
					\end{itemize}					 
				\item \textbf{9 mode dataset}: 9 Zernike mode coefficients, their RMSE in the range:
					\begin{itemize}
						\item Modes 2 and 3: [-0.22, 0.22]
						\item Modes 4, 5 and 6: [-0.22, 0.22]
						\item Modes 7, 8, 9 and 10: [-0.22, 0.22]
					\end{itemize}					 
				\item \textbf{14 mode dataset}: 14 Zernike mode coefficients, their RMSE in the range:
					\begin{itemize}
						\item Modes 2 and 3: [-0.142, 0.142]
						\item Modes 4, 5 and 6: [-0.142, 0.142]
						\item Modes 7, 8, 9 and 10: [-0.142, 0.142]
						\item Modes 11, 12, 13, 14 and 15: [-0.142, 0.142]
					\end{itemize}					 
				\item \textbf{20 mode dataset}: 20 Zernike mode coefficients, their RMSE in the range:
					\begin{itemize}
						\item Modes 2 and 3: [-0.1, 0.1]
						\item Modes 4, 5 and 6: [-0.1, 0.1]
						\item Modes 7, 8, 9 and 10: [-0.1, 0.1]
						\item Modes 11, 12, 13, 14 and 15: [-0.1, 0.1]
						\item Modes 16, 17, 18, 19, 20 and 21: [-0.1, 0.1]
					\end{itemize}					 
				\item \textbf{27 mode dataset}: 27 Zernike mode coefficients, their RMSE in the range:
					\begin{itemize}
						\item Modes 2 and 3: [-0.07, 0.07]
						\item Modes 4, 5 and 6: [-0.07, 0.07]
						\item Modes 7, 8, 9 and 10: [-0.07, 0.07]
						\item Modes 11, 12, 13, 14 and 15: [-0.07, 0.07]
						\item Modes 16, 17, 18, 19, 20 and 21: [-0.07, 0.07]
						\item Modes 22, 23, 24, 25, 26, 27 and 28: [-0.07, 0.07]
					\end{itemize}					 
				\item \textbf{35 mode dataset}: 35 Zernike mode coefficients, their RMSE in the range:
					\begin{itemize}
						\item Modes 2 and 3: [-0.05, 0.05]
						\item Modes 4, 5 and 6: [-0.05, 0.05]
						\item Modes 7, 8, 9 and 10: [-0.05, 0.05]
						\item Modes 11, 12, 13, 14 and 15: [-0.05, 0.05]
						\item Modes 16, 17, 18, 19, 20 and 21: [-0.05, 0.05]
						\item Modes 22, 23, 24, 25, 26, 27 and 28: [-0.05, 0.05]
						\item Modes 29, 30, 31, 32, 33, 34, 35 and 36: [-0.05, 0.05]
					\end{itemize}					 
				\item \textbf{44 mode dataset}: 44 Zernike mode coefficients, their RMSE in the range:
					\begin{itemize}
						\item Modes 2 and 3: [-0.04, 0.04]
						\item Modes 4, 5 and 6: [-0.04, 0.04]
						\item Modes 7, 8, 9 and 10: [-0.04, 0.04]
						\item Modes 11, 12, 13, 14 and 15: [-0.04, 0.04]
						\item Modes 16, 17, 18, 19, 20 and 21: [-0.04, 0.04]
						\item Modes 22, 23, 24, 25, 26, 27 and 28: [-0.04, 0.04]
						\item Modes 29, 30, 31, 32, 33, 34, 35 and 36: [-0.04, 0.04]
						\item Modes 37, 38, 39, 40, 41, 42, 43, 44 and 45: [-0.04, 0.04]
					\end{itemize}					 
			\end{itemize}
			
	\subsection{PSFs intensities dataset}
		8 datasets of 5000 PSF intensities are created from the 5 zernike coefficients dataset, each datapoint being a 128x128 matrix
		
	\subsection{LP mode coefficients dataset}
		8 datasets of 5000 LP coefficients are created from the 5 zernike coefficients dataset, each datapoint being a 19x2 matrix dividing the real and imaginary part of the LP coefficients
		
	\subsection{Output fluxes dataset}
		The output fluxes are obtained using the 19 mode PL transfer matrix.
		8 datasets of 5000 Output fluxes are created from the 5 zernike coefficients dataset, each datapoint being a 19x1 vector
		
\part{Informe de prácticas I}

\section{Estancia en el IAC}
En la estancia del IAC escribí apuntes con los conceptos que estudié sobre telescopios, óptica y corrección de frente de onda que se pueden ver en el pdf de \textit{Basic concepts.pdf}\\

También hice un algoritmo genético para organizar las guardias del telescopio que está en el \href{https://github.com/Light-Bridges/fellows}{repositorio github de LB}.

\section{La linterna fotónica}
Es el dispositivo de fibras ópticas que hemos estado utilizando en SAIL como sensor de frente de onda.\\

La linterna fotónica tiene en un extremo una fibra óptica multimodo y en el otro, múltiples fibras unimodo con las que codifica la amplitud y la fase del frente de onda en la intensidad de cada una de las fibras. Para más información ver el pdf de \textit{Basic concepts.pdf}

\section{Prototipo de red neuronal para reconstrucción del frente de onda}
El primer proyecto con código que desarrollé fue un prototipo de la red neuronal que reconstruiría el frente de onda dadas las fases de un subconjunto de puntos del frente de onda. Se puede ver en este \href{https://github.com/Dacarpe03/SurfaceReconstruction}{repositorio de github}

\section{Reconstrucción de frente de onda con datos de laboratorio}
Con datos generados con un SLM y los outputs de la linterna fotónica entrené los primeros modelos de reconstrucción en este \href{https://github.com/Dacarpe03/PLImageReconstruction}{repositorio}. Se puede ver en la parte 1 del pdf \textit{SAIL.pdf}

\section{Reconstrucción de frente de onda con datos de simulación}
Con datos generados con HCipy y los outputs de la linterna fotónica entrené modelos de reconstrucción de frente de onda más avanzados en este \href{https://github.com/Dacarpe03/PLImageReconstruction}{repositorio}. Se puede ver en la parte 2 del pdf \textit{SAIL.pdf}. Además hice una base de datos para guardar los resultados y arquitecturas de las redes neuronales en este \href{https://github.com/Dacarpe03/SAILExperimentsDatabase}{repositorio}. 

\section{Medición de cantidad de información almacenada}
Con teoría de la información determinamos la relación entre el PSF y el output de la linterna fotónica en este \href{https://github.com/Dacarpe03/PLImageReconstruction}{repositorio}. Se puede ver en las partes 3-8 del pdf \textit{SAIL.pdf}.